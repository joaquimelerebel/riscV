%l'espace interligne
\usepackage{setspace}
%les images
\usepackage{graphicx}
	\graphicspath{ {../} }
 %tableaux multipages
\usepackage{longtable}
%pour gerer les flotant -> images
\usepackage{float}
%séparer la page en 2
\usepackage{blindtext}
%tableaux pour les bytefields
\usepackage{bytefield}
%tableaux à pleins de colones
\usepackage{multicol}
% avoir des tableaux et des images propres en multicols
\usepackage{wrapfig}
%theorical computer science symbols
\usepackage{stmaryrd}
%little lines on the tables
\usepackage{booktabs}
%ajoute les slash dans les tables
\usepackage{diagbox}
%indent after a section title
\usepackage{indentfirst}
%use of boxes ... encadrer du text
\usepackage{awesomebox}
%ajout des emojis
\usepackage{emoji}
%maths symbols
\usepackage{amssymb}
\usepackage{amsmath}
	\newcommand{\Lapl}{\mathop{\mathcal{L}}}
\usepackage{mathtools}
    \DeclarePairedDelimiter\floor{\lfloor}{\rfloor}
%avoir du UTF8
\usepackage[utf8]{inputenc}
	\pagenumbering{arabic}
%police francais
\usepackage[T1]{fontenc}
%Avoir Tables de matiere en fr
\usepackage[french,english]{babel}
%insertion pdf
\usepackage{pdfpages}
%add the refs
\usepackage[style=ieee]{biblatex}
\addbibresource{ref.bib}
%\addbibresource{res.bib}
%couleurs
\usepackage{xcolor}
	\definecolor{codegreen}{rgb}{0,0.6,0}
	\definecolor{codegray}{rgb}{0.5,0.5,0.5}
	\definecolor{codepurple}{rgb}{0.58,0,0.82}
	\definecolor{backcolour}{rgb}{0.95,0.95,0.92}


    \definecolor{mGreen}{rgb}{0,0.6,0}
    \definecolor{mGray}{rgb}{0.5,0.5,0.5}
    \definecolor{mPurple}{rgb}{0.58,0,0.82}
    \definecolor{backgroundColour}{rgb}{0.95,0.95,0.92}

%avoir des listes
\usepackage{listings}
% RISC-V Assembler syntax and style for latex lstlisting package
% 
% These are risc-v commands as per our university (University Augsburg, Germany) guidelines.
%
% Author: Anton Lydike
%
% This code is in the public domain and free of licensing

% language definition
\lstdefinelanguage[RISC-V]{Assembler}
{
  alsoletter={.}, % allow dots in keywords
  alsodigit={0x}, % hex numbers are numbers too!
  morekeywords=[1]{ % instructions
    lb, lh, lw, lbu, lhu,
    sb, sh, sw,
    sll, slli, srl, srli, sra, srai,
    add, addi, sub, lui, auipc,
    xor, xori, or, ori, and, andi,
    slt, slti, sltu, sltiu,
    beq, bne, blt, bge, bltu, bgeu,
    j, jr, jal, jalr, ret,
    scall, break, nop
  },
  morekeywords=[2]{ % sections of our code and other directives
    .align, .ascii, .asciiz, .byte, .data, .double, .extern,
    .float, .globl, .half, .kdata, .ktext, .set, .space, .text, .word
  },
  morekeywords=[3]{ % registers
    zero, ra, sp, gp, tp, s0, fp,
    t0, t1, t2, t3, t4, t5, t6,
    s1, s2, s3, s4, s5, s6, s7, s8, s9, s10, s11,
    a0, a1, a2, a3, a4, a5, a6, a7,
    ft0, ft1, ft2, ft3, ft4, ft5, ft6, ft7,
    fs0, fs1, fs2, fs3, fs4, fs5, fs6, fs7, fs8, fs9, fs10, fs11,
    fa0, fa1, fa2, fa3, fa4, fa5, fa6, fa7
  },
  morecomment=[l]{;},   % mark ; as line comment start
  morecomment=[l]{\#},  % as well as # (even though it is unconventional)
  morestring=[b]",      % mark " as string start/end
  morestring=[b]'       % also mark ' as string start/end
}

% usage example:

% define some basic colors
\definecolor{mauve}{rgb}{0.58,0,0.82}

\lstdefinestyle{riscv}{
    language=[RISC-V]{Assembler}, 
    backgroundcolor=\color{white}, 
    numberstyle=\tiny\color{codegray},
    stringstyle=\color{white},
    basicstyle=\footnotesize\sffamily\color{black},
    breakatwhitespace=false,         
    breaklines=true, 
    rulecolor=\color{black},
    captionpos=b,                    
    keepspaces=true,                 
    numbers=none,                    
    numbersep=5pt,                  
    showspaces=false,                
    showstringspaces=false,
    showtabs=false,                  
    tabsize=2,
    escapeinside={<@}{@>}
}

\lstdefinestyle{console}{
        backgroundcolor=\color{black}, 
        numberstyle=\tiny\color{codegray},
        stringstyle=\color{white},
        basicstyle=\tiny\sffamily\color{white},
        breakatwhitespace=false,         
        breaklines=true, 
        rulecolor=\color{black},
        captionpos=b,                    
        keepspaces=true,                 
        numbers=none,                    
        numbersep=5pt,                  
        showspaces=false,                
        showstringspaces=false,
        showtabs=false,                  
        tabsize=2
    }
\lstdefinestyle{consoleLarge}{
    backgroundcolor=\color{black}, 
    numberstyle=\tiny\color{codegray},
    stringstyle=\color{white},
    basicstyle=\footnotesize\sffamily\color{white},
    breakatwhitespace=false,         
    breaklines=true, 
    rulecolor=\color{black},
    captionpos=b,                    
    keepspaces=true,                 
    numbers=none,                    
    numbersep=5pt,                  
    showspaces=false,                
    showstringspaces=false,
    showtabs=false,                  
    tabsize=2
}
    
%liens/references	
\usepackage{hyperref}
	\hypersetup{
    		colorlinks=true,
    		linkcolor=blue,
    		filecolor=magenta,
    		urlcolor=cyan,
	}
%captions -> descriptions d'elements
\usepackage{caption}
\usepackage{subcaption}
%tableaux
\usepackage{array,multirow,makecell}
	\setcellgapes{1pt}
	\makegapedcells
	\newcolumntype{R}[1]{>{\centering\arraybackslash }b{#1}}
	\newcolumntype{L}[1]{>{\centering\arraybackslash }b{#1}}
	\newcolumntype{C}[1]{>{\centering\arraybackslash }b{#1}}
%lipsum pour faire du template
\usepackage{lipsum}
%les marges et formats de pages
\usepackage{geometry}
 	\geometry{a4paper,
 			total={170mm,257mm},
 			left=20mm,
 			top=20mm,}
%s'assurer qu'on a bien les section en roman
\renewcommand{\thesection}{\Roman{section}}
\renewcommand{\thesubsection}{\thesection.\Roman{subsection}}
%fonction pour faire une division de la page en 3 horizontales
\newcommand\textline[4][t]{%
  \par\smallskip\noindent\parbox[#1]{.5\textwidth}{\raggedright#2}%
  \parbox[#1]{.5\textwidth}{\raggedleft{#3}}\par\smallskip%
}

\makeatletter
  \renewcommand\l@section{\@dottedtocline{2}{1.5em}{3em}}
  \renewcommand\l@subsection{\@dottedtocline{2}{3em}{5em}}
  \renewcommand\l@subsubsection{\@dottedtocline{2}{6em}{7em}}
\makeatother

\usepackage[newfloat]{minted}
\newenvironment{code}{\captionsetup{type=listing}}{}
\SetupFloatingEnvironment{listing}{name=Source Code}
%schémas 
\usepackage{tikz}
\usetikzlibrary{positioning,shapes,shadows,arrows,intersections}
\usetikzlibrary{positioning}
\usetikzlibrary{decorations.text}
\usetikzlibrary{decorations.pathmorphing}
\usepackage{pgfplots}
\usepackage{circuitikz}
%\tikzstyle{arduidev}=[draw, text width=6em, minimum height=8em]
